\documentclass{article}
\usepackage{xcolor}
\usepackage{ulem}


\definecolor{styC}{rgb}{1,0,1}    %magenta
\definecolor{nsrC}{HTML}{0F8282}  %teal
\definecolor{noteC}{rgb}{0,0,1}   %blue
\definecolor{todoC}{HTML}{FE5010} %orange

\newcommand*{\TODO}[2][todoC]{{\color{#1} #2}}
\newcommand*{\TODOsty}[2][styC]{\TODO[#1]{#2}}
\newcommand*{\TODOnsr}[2][nsrC]{\TODO[#1]{#2}}
\newcommand*{\NOTE}[2][noteC]{\TODO[#1]{\textit{#2}}}
\newcommand*{\TODOfn}[2][noteC]{\TODO[#1]{[\footnote{\TODO[#1]{#2}}]}}
\newcommand*{\TODOref}[2][todoC]{\TODOfn[#1]{\textbf{BIB-REF}: #2}}

\usepackage{fancyhdr}
\usepackage{flushend}
\usepackage{amsfonts,amssymb,amsmath,alltt}
\usepackage{stmaryrd}
\usepackage{xspace}
\usepackage{graphicx}

\usepackage{makeidx}
\pagestyle{plain}
\usepackage[utf8]{inputenc}
\usepackage{url}
\usepackage[english]{babel}

\begin{document}

%%%%%
\section{Legend}
Dear Florian,

here's how to read this.\medskip\\
%%

I use the following color code to indicate different levels of incompleteness or uncertainty (usually with regard to the language, not the facts):
\begin{itemize}
\item \TODOnsr{\underline{Not sure}}\\
  Doesn't sound weird exactly but some doubt lingers and I'm simply not sure
  (punctuation, grammar, wording ...).
\item \TODOsty{\underline{Pretty sure it's wrong or could be better, at least}}\\
  Might be poor style or I don't know which word to pick or something else
  that feels not quite right/appropriate.
\item \NOTE{\underline{Comments}}\\
  Notes, remarks and such like
\item \TODO{\underline{TODO}}\\
  Something to check or fix. Mostly bibtex references (see footnotes starting with ``\TODO{\textbf{BIB-REF}:}'').
\end{itemize}

If there's anything to say beyond the color, I put it in a footnote. These footnotes are colored the same way the corresponding phrase is and bracketed to set them apart from ordinary footnotes -- there's indeed at least one of those hidden somewhere ;D...

Section headings try to summarize what a passage is (mainly) concerned with -- think of them as kind of working titles.
The sections are in no particular order.\medskip\\

Hope you can get anything out of it -- in parts it's rather notes than ready-to-use texts (and less than I expected/hoped for) :/

Use it, call me or send me an email in case you have any questions, send me a picture of scribbles you put on a print out and I'll send you (hopefully) ready-to-use revisions of stuff you want to use, whatever suits you ... just let me know :)\medskip\\

LG\bigskip\\
PS: I honestly don't think the development of the German corona app took remarkably long -- I seem to remember reading something like that in your introduction.
Judging from my daily (professional) experience I'm sort of
amazed they got it done so \textit{fast}, i.e. almost in time ;) -- sadly enough, I really mean that...
I don't doubt it was costly, though.


%%%%%
\section{Related works: DP-3T and PEPP-PT}
%
This paper \TODOnsr{is mainly concerned} with the \TODOnsr{protocol architecture} proposed by the \textit{Decentralized Privacy-Preserving Proximity Tracing} project (DP-3T).
The main reason to focus \TODOnsr{on} this particular family of protocols is the \textit{Exposure Notification Framework}, \TODOsty{jointly published} by Apple and Google, \TODOsty{following/that follows} a number of core principles of the DP-3T proposal. This API is not only utilized \TODOsty{in/by} the German \textit{Corona-Warn-App} (CWA) but has the potential of being widely adopted in future app developments \TODOsty{(we might see/that might emerge)}, due to the reach of \TODOsty{players like Apple and Google}.

There are, however, competing architectures noteworthy, namely protocols developed under the roof of the \textit{Pan-European Privacy-Preserving Proximity Tracing} project (PEPP-PT), e.g. PEPP-PT-ROBERT\TODOref{ROBERT github repository} and PEPP-PT-NTK\TODOref{NTK ???; mentioned in ``Response to Analysis of DP-3T''}.

Neither DP-3T nor PEPP-PT \TODOnsr{is synonym for} just a single protocol. Each project endorses different protocols with unique properties in terms of privacy and data protection.\\
Yet, on \TODOsty{a more abstract level/a higher level of abstraction}, it seems \TODOnsr{feasible} to distinguish two basic architectures: Protocols \TODOnsr{as} endorsed by PEPP-PT might be characterized as centralized architectures whereas DP-3T-inspired protocols follow a (more) decentralized approach.\footnote{%
  \TODOsty{As we will see}, DP-3T involves a central backend server. It is decentralized with regard to \TODOsty{collection and evaluation} of contact information:
  In centralized architectures the server provides a risk scoring services, whereas decentralized approaches rely on local risk assessment and, thus, do not need to share contact information with the \TODOnsr{backend}.}

\NOTE{%
  \vspace{.25em}\\
  ``DP-3T is a free-standing effort, originally started at EPFL and ETHZ.''\\
  Some of its members participated in the PEPP-PT project but have since
  resigned from the initiative.\TODOfn[noteC]{%
    github:DP-3T/documents\#april-8th-2020-the-relationship-between-dp-3t-and-pepp-pt
  }}

%%%%%
\section{Basic DP-3T protocol: How does it work? How does it relate to ENF/CWA? (broad strokes)}
%
Upon installation\TODOnsr{,} the app generates secret daily seeds to derive so called \textit{Ephemeral IDs} (EphIDs) \TODOnsr{from}. EphIDs are generated locally with cryptographic methods and cannot be connected to one another but only reconstructed \TODOnsr{from} the \TODOsty{secret seed they were derived from/corresponding secret seed}.\\
During normal operation each client broadcasts \TODOnsr{their} EphIDs via Bluetooth \TODOsty{whilst}\TODOfn[styC]{Too pompous? Is ``while scanning...'' better?} scanning \TODOnsr{for} EphIDs broadcasted by other devices in the vicinity. Collected EphIDs are stored locally along with associated metadata such as signal attenuation and date. In DP-3T the contact information gathered is never shared but only evaluated \TODOsty{locally/on the device}.\\
After patients are diagnosed \TODOnsr{(officially)}\TODOnsr{,} they are entitled to upload specific data to a central backend server. This data is accumulated by the \TODOnsr{backend}\TODOfn[nsrC]{Is ``backend'' a valid ellipsis or is it ``backend server''?} and redistributed to all clients regularly to \TODOnsr{provide the means for} local risk \TODOsty{scoring/assessment}, i.\,e. determining whether collected EphIDs match \TODOnsr{those} broadcasted by \TODOnsr{now-confirmed} COVID-19 patients during the last \TODOsty{e.\,g. 14}\TODOfn[styC]{%
      I can't figure out how to express that it is some fixed number of days that might be 14 but doesn't have to be (without a lot of lengthy blahblah or getting all formal using some $N$ and ``where ...'').} %
days.

In the most simple (and insecure) protocol proposed by DP-3T \TODOref{Low-cost decentralized proximity tracing; DP-3T Whitepaper p. 14ff} this \TODOsty{basically} translates into publishing the daily seeds used to derive EphIDs \TODOnsr{from}.
\TODOsty{Aside from additional security measures}\TODOfn[styC]{In short: I want to name it but don't talk about it (since we're not really concerned with this level of detail). But this sounds just sh...} like signing content the server provides \TODOnsr{as a general rule}, the protocol implemented by ENF and, hence, CWA follows this low-cost design.\TODOref{CWA solution architecture ($\rightarrow$ github:CWA); probably ENF specification} DP-3T proposes two other, more sophisticated protocols that improve \TODOsty{privacy and data protection properties} to different degrees but are more \TODOsty{costly/complicated}\TODOfn[styC]{``complicated'' sounds lame but ``costly'' is, perhaps, to simple/literal} to set up.

\NOTE{%
  \vspace{.25em}\\
  I have a nice drawing to illustrate this (including some possible mitigation techniques outlined by CWA and DP-3T, respectively).
  I'll scan it when I'm in the office this week.}


%%%%%
\section{Formal verification using HOL: Why bother?}
%
\NOTE{%
  Sorry, loose notes is all I got on this one...}
%

\begin{itemize}
\item Quote\TODOref{DP-3T Whitepaper p. 2}:
  ``Such a design builds on strong, mathematically provable support for privacy and data protection goals [...]''\smallskip\\
  \NOTE{DP-3T on their design and how wisely chosen it is ;) We have to use this! Something along the lines of:}
  Despite strong claims with regard to mathematical support, there is, as of yet, no formal verification (known to the authors). \TODOsty{So we just thought, we throw our hat in the ring.}
\item Explore different concepts of refinement: Maybe it is possible to pin down (i.e. exemplify) different concepts of refinement (data refinement, action refinement, trace refinement (aka spec refinement?) ...) -- and perhaps combinations thereof -- with concrete attack scenarios. I'm sure there'd be something to learn from that (esp. with respect to IsabelleAT).\\
  \NOTE{Sorry, can't specify it any further :(}
\end{itemize}


%%%%%
\section{What about attacks? Which of them do we consider? \sout{Why is that?}}
%
There is a variety of \TODOsty{interesting/noteworthy} privacy and security issues that \TODOsty{justify/deserve} \TODOsty{formal consideration} (and verification). The debate emerging between advocates of centralized architectures and those in favor of a decentralized approach in particular yields a lot of \TODOsty{interesting material} in terms of attack scenarios and possible mitigation strategies, e.g. \TODOref{Analysis of DP-3T ($\rightarrow$ github:PEPP-PT); Response ($\rightarrow$ github:DP-3T); PSRE and (perhaps) Analysis of PEPP-PT by DP-3T (somewhere on github:DP-3T, I guess; links can be found in Response to Analysis of DP-3T)}.

\NOTE{\\%
  ``Analysis of DP-3T'' for example addresses the following attack scenarios
  (categories of attacks):
  \begin{itemize}
  \item False Alert Injection Attacks
    \begin{itemize}
    \item Backend Impersonation / False Report
    \item Replay and Relay attacks
    \end{itemize}
  \item \TODOsty{Deanonymization/Tracking}\TODOfn[styC]{%
    Not sure you could really call it ``Deanonymization attack''. I'm not even sure ``deanonymization'' is a proper word...}
    Attacks (the authors don't call it that)
    \begin{itemize}
    \item Based on establishing a connection between collected EphIDs and
      (target) users, an attacker might:
      \begin{itemize}
      \item Track user's positions by tracking their Bluetooth signal
        (as recognized by the EphIDs they broadcast).\\
        $\rightarrow$ \textit{Location Tracking}
      \item Disclose a user's (positive) test status.\\
        $\rightarrow$ \textit{Test Status Disclosure}
      \item Disclose private encounters.
      \end{itemize}
    \end{itemize}
  \end{itemize}
}

\TODOsty{We focus on what we call Deanonymization Attacks.}\medskip\\

Although addressed in ``Analysis of DP-3T'', the two attacks we're concerned with, are classified by DP-3T as generic and inherent risks, respectively\TODOref{Privacy and Security Risk Evaluation of Digital Proximity Tracing Systems: GR 5 and IR 1}, \sout{and, hence, only \TODO{briefly} \TODOsty{addressed}} \TODO{$\leftarrow$ no they're not:} \NOTE{PSRE actually contains a quite thorough analysis of how infected individuals might be identified. (With pretty much the same conclusion I drew.)\\
  \label{PSRE}
  I only realized yesterday, what a treasure trove PSRE really is
  (or would have been)...
  Most (if not all) of the stuff I was thinking about the last couple of weeks
  is noted in there... Apparently I didn't read the right document :/ ...
  To be honest, this is a bit discouraging...\\
  See: Privacy and Security Risk Evaluation of Digital Proximity Tracing
  Systems, p. 5f -- IR 1: Identifying infected individuals}


%%%
\subsection{Why choose Bluetooth Beacons as starting point for building a formal representation?}
\begin{itemize}
\item Simple and thus easy to formalize.
\item Deanynomization attacks start here.
\item The Bluetooth Low Energy Beacon mechanism is employed in DP-3T and PEPP-PT protocols \TODOnsr{alike}.\\
  \NOTE{Perhaps the only thing they can agree on ;)}\smallskip\\
  There is one significant difference, though:
  \begin{itemize}
  \item DP-3T:
    The architecture requires that reported users can be recognized
    by the apps of their encounters
    (to some extend; e.\,g. mitigation by cuckoo filters).\smallskip\\
    Quote \TODOref{Analysis of DP-3T, p. 10}:
    ``Based on that, the only way to mitigate this attack is to deny the same
    privileges as the apps to individuals.''\smallskip\\
    \NOTE{(IMHO) peculiar English; even so, a direct quote}
  \item PEPP-PT:
    The backend server issues EphIDs and never shares any secret seeds,
    i.\,e. deanynomizing EphIDs \TODOnsr{arguably} requires more effort
    (= breaking into the server).
    \TODOnsr{In return}, more sensitive data (= contact time data)
    is shared with the server\TODOnsr{, to begin with}.
  \end{itemize}
\end{itemize}

%%%
\subsection{Deanonymization Attacks}
%%
\subsubsection{Different goals, different attacks, different levels of mitigation}
Regularly changing EphIDs might prevent location tracking effectively, or, \TODOnsr{at least, limit it to an acceptable degree}. While \TODOsty{(possibly)} having \TODOnsr{little to no effect}\TODOfn[nsrC]{Maybe that's too much; better (?): ``significantly/much less impact/effect on''} \TODOsty{on/in} scenarios where \TODOsty{the goal is to learn if} some particular user was infected.

\NOTE{%
  \vspace{.25em}\\
  \underline{Idea:}\\
  In order for location tracking to succeed, a valid/active EphID (i.\,e. one
  that is actually/currently broadcasted) is required.
  As a consequence, the time frame in which (additional) location information
  can be gathered is limited by the frequency at which EphIDs change.\\
  A positive test result, on the other hand, can be reported a (fixed)
  maximum number of days after the compromised EphID was used
  and still yield the desired information.
  This maximum time frame is independent of any frequency at which EphIDs
  may change.}

\NOTE{%
  \vspace{.25em}\\
  I have two (pretty high level, similar -- i.\,e. comparable) drawings
  depicting Location Tracking and Test Status Disclosure, respectively.\\
  Could help to illustrate this
  ``strategy X works in this context pretty good,
  but in that one not so much'' idea.}

%%
\subsubsection{Tracking someone's location}
\TODOnsr{If} an attacker can connect EphIDs to specific users or devices, he can track their whereabouts: For as long as a deanonymized EphID is valid, i.\,e. is broadcasted by the respective client, scanning for this id would suffice to locate \TODOnsr{this} client.\\
The range of Bluetooth reception can be improved by \TODOsty{sufficiently good/ high-performance} antennas. While \TODOnsr{stationary set up} hardware \TODOsty{can free an attacker of the need for actual, physical presence}.

%%
\subsubsection{Revealing someone's (positive) test status}
Like any notification about \TODOsty{e.\,g. a STD}\TODOfn[styC]{%
  Again, I don't know if you can use ``e.\,g.'' like that}
  might inform a spouse about the infidelity of the other, given the right circumstances, being notified about a possible COVID-19 exposure might reveal illicit information to the recipient like who it was \TODOsty{that} tested \TODOnsr{positive}.\\
  This is an inherent vulnerability of any such \TODOnsr{(notification)} system\TODOnsr{,
    whether it is analogous or} \TODOsty{digital/digitally aided}.
  The likelihood of said circumstances to \TODOnsr{occur}, however, might increase when notifications where carried out automatically.\\
  This is a \TODOnsr{quite interesting question to investigate}; its formal treatment could benefit considerably from the flexibility IsabelleAT offers: The ability to incorporate aspects of the physical world is \TODOsty{surely of special interest in this context}\TODOfn[styC]{Lame and awkward!}.

  \NOTE{\smallskip\\%
    DP-3T:PSRE\TODOref[noteC]{%
      Privacy and Security Risk Evaluation of Digital Proximity Tracing Systems:
      IR 1, p. 5f}
    contains a detailed analysis of how infected individuals might be identified.
    In the end, however, this attack --
    being (rightfully) classified as inherent risk -- is dismissed:
    ``These weaknesses are both minor and a generic problem in all
    designs based on Bluetooth proximity detection.''\TODOref[noteC]{%
      Response to Analysis of DP-3T:
      I: s.5.2 Deanonymizing Known Reported User, p. 2}\medskip\\
    If there was a chance of quantifying quality of support
    (i.\,e. a meaningful, yet abstract concept of effectiveness
    of proximity tracing in terms of aiding traditional contact tracing)...
    The question of ``cost'' (e.\,g. in the sense of increased risk)
    could be interesting, indeed.\\
    Too bad, this quantification business seems rather unlikely.}
  \pagebreak\\
  Quote \TODOref{%
    Privacy and Security Risk Evaluation of Digital Proximity Tracing Systems:
    IR 1, p. 6}:
  ``We emphasize that these attacks work against any contact tracing system,
  as they rely on the core proximity tracing functionality:
  notifying at-risk people.
  Without this notification, proximity tracing system would be useless.
  The risk is inherent to proximity tracing.''\medskip\\
  %
  Suppose, however, proximity tracing could be restricted to places
  of particular interest, e.\,g. places where ``mass gatherings'' are
  more likely to occur:
  public transport, supermarkets and such.\\
  \TODOsty{Maybe (probably?) this would be worse, privacy-wise,
    maybe it would be beneficial.}
  In either way, such a system wouldn't be useless.\\
  How much usefulness really stands to loose
  (by any digital proximity tracing system) is a question
  that still needs answering anyway.
  \NOTE{(As far as I know -- but that's not very far, to be honest)}
\end{document}
