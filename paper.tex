\documentclass{llncs}

\usepackage{fancyhdr}
\usepackage{flushend}
\usepackage{amsfonts,amssymb,amsmath,alltt}
\usepackage{stmaryrd}
\usepackage{xspace}
\usepackage{graphicx}
\usepackage{color}

\usepackage{makeidx}
\pagestyle{plain}
\usepackage[utf8]{inputenc}
\usepackage{url}
\usepackage[english]{babel}


\renewcommand{\ttdefault}{cmtt}
%This version DOES NOT suppress line breaks
% newcommands etc
\newenvironment{ttbox}{\begin{alltt}\ttbraces\small\tt}%
                      {\end{alltt}}
%the new definition of \. suppresses line breaks
\def\ttbraces{\let\.=\nobreak\chardef\{=`\{\chardef\}=`\}\chardef\|=`\\}

\newcommand{\symb}[1]{\makebox{\it #1}} 
\newcommand\ie{i.e.\!\,, }
\newcommand\rel{Re${\cal{L}}$}
\newcommand{\TODO}[1]{\textcolor{red}{\textbf{[TODO:#1]}}}
\newcommand\ttand{\mbox{{$\land$}}}
\newcommand\ttor{\mbox{{$\lor$}}}
\newcommand\ttcap{\mbox{{$\cap$}}}
\newcommand\ttcup{\mbox{{$\cup$}}}
\newcommand\ttfun{\mbox{{$\Rightarrow$}}}
\newcommand\ttmimp{\mbox{{$\Longrightarrow$}}}
\newcommand\ttimp{\mbox{{$\longrightarrow$}}}
\newcommand\ttequiv{\mbox{{$\equiv$}}}
\newcommand\ttexists{\mbox{{$\exists$}}}
\newcommand\ttforall{\mbox{{$\forall$}}}
\newcommand\ttneg{\mbox{{$\neg$}}}
\newcommand\ttneq{\mbox{{$\neq$}}}
\newcommand\ttin{\mbox{{$\in$}}}
\newcommand\ttnin{\mbox{{$\notin$}}}
\newcommand\ttImp{\mbox{{$\Longrightarrow$}}}
\newcommand\ttlam{\mbox{\( \lambda \)}}
\newcommand\tttimes{\mbox{\( \times \)}}
\newcommand\ttlbrack{\mbox{\(\llbracket\)}}
\newcommand\ttrbrack{\mbox{\( \rrbracket \)}}
\newcommand\noie{\textit{i.e.},\xspace}
\newcommand\eg{~\textit{e.g.},\xspace}
\newcommand\noeg{\textit{e.g.},\xspace}
\newcommand\ttatI{\mbox{\( @_G \)}}
\newcommand\ttto[1]{\mbox{{$\to^{#1}$}}}
\newcommand\ttleq{\mbox{{$\le$}}}
\newcommand\ttrelI{\mbox{{$\to_{i}$}}}
\newcommand\ttalpha{\mbox{{$\alpha$}}}
\newcommand\tttau{\mbox{{$\tau$}}}
\newcommand\ttsubseteq{\mbox{{$\subseteq$}}}
\newcommand\ttf{\mbox{{$f$}}}
\newcommand\ttvdash{\mbox{{$\vdash$}}}
\newcommand\ttref[1]{\mbox{\(\sqsubseteq_{#1}\)}}
\newcommand\ttNatt{\mbox{{$\mathcal{N}$}}}
\newcommand\ttattand[1]{\mbox{{$\oplus_{\wedge}^{#1}$}}}
\newcommand\ttattor[1]{\mbox{{$\oplus_{\vee}^{#1}$}}}
\newcommand\ttrelIstar{\mbox{{$\to_{i}^*$}}}


\begin{document}
\frontmatter
  
\mainmatter
\title{Modeling and analyzing the Corona-virus warning app with the Isabelle Infrastructure framework}
\author{Florian Kamm\"uller and Bianca Lutz}

\institute{Middlesex University London and\\ Technische Universit\"at Berlin\\
\email{f.kammueller@mdx.ac.uk|bialut@gmail.com}
}
\maketitle
\begin{abstract}
We provide a model in the Isabelle Infrastructure framework of the recently published
Corona-virus warning app. The app supports breaking infection chains by informing users
whether they have been in close contact to an infected person. The app has a decentralized
architecture that supports anonymity of users.
We provide a formal model of the existing app with the Isabelle Infrastructure framework
to show up some natural attacks in a very abstract model. We then use the security
refinement process of the Isabelle Infrastructure framework to highlight how the use of
continuously changing ephemeral ids improves the anonymity.
\end{abstract}

\section{Introduction}
\label{sec:intro}
The German Chancellor Angela Merkel has strongly supported the publication of
the mobile phone Corona warning app by publicly proclaiming that the ``Corona
App deserves your trust'' \cite{bundes:20}. Many millions of mobile phone users
in Germany have downloaded the app with 6 million on the first day.
This app is one amongst many similar application that aim at the very important goal
to ``break infection chains'' by providing timely information of users whether they
have been exposed to close contact with a person that has been infected.

The Corona-virus warning app has taken a long time to develop being published only on
16th June 2020. It was a quite costly project but this was mainly due to the management of
Telekom and SAP being in the driving seat. But the app has been designed with great
attention on privacy:  a distributed architecture \cite{} has been adopted after a long and
heated debate with supporters of a central architecture. The distributed architecture is
based on a very clever distributed application design whereby users phones are sending
highly anonymized so called ``Ephemeral IDs'' at physical locations via the Bluetooth
protocol. The app saves those IDs of people in close proximity. When at a later date an
infected person reports his infection to a central server, the unique root ID is published
and in the daily check all mobile phones connecting to the central server can download
the root IDs of infected people. Since the Ephemeral IDs can be mapped to the root ID
all Ephemeral IDs that have been saved over the last 14 days allow users phones to
regularly check whether their user has been exposed to an infected person and issue
a warning to the user. The warning issued by the Corona warning app entitles to
having a Corona test done (which at the time of writing is not normally possible).

The Isabelle Infrastructure framework \cite{kam:20a} allows modeling and analyzing
architecture and scenarios including physical and logical entities, actors, and policies
within the interactive theorem prover Isabelle supported with temporal logic, Kripke
structures, and attack trees. It has been applied for example to Insider analysis in
airplanes \cite{kam:20b}, privacy in IoT healthcare \cite{kam:18b}, and recently also
to blockchain protocols \cite{kn:20}.


\TODO{Motivation: Why bother re-engineering a formal specification for a nicely developed privacy oriented app? physical aspects (adoption rate), formal proof etc}

In this paper, we first provide some background in Section \ref{sec:background}:
we give a detailed overview of the development history and security and privacy
relevant parts of the Corona-virus warning app (Section \ref{sec:history})
and some essential facts about the Isabelle Infrastructure framework
(Section \ref{sec:isainf}).
We then present our model (Section \ref{sec:model}) and analysis of privacy and
attacks (Section \ref{sec:ana}) before drawing some conclusions (Section \ref{sec:concl}).

The formal model in the Isabelle insider framework is fully mechanized and proved in
Isabelle (sources available \cite{kam:18smc}). 

\section{Background}
\label{sec:background}

\subsection{History of Decentralized Architecture}
\label{sec:history}

\subsection{Isabelle Infrastructure framework}
\label{sec:isainf}


\section{Model}
\label{sec:model}

\section{Analysis}
\label{sec:ana}

\section{Conclusions}
\label{sec:concl}

\bibliographystyle{abbrv}
\bibliography{insider}

\end{document}
